\documentclass[a4paper, 10pt]{letter}
\usepackage{fullpage}

\name{Johann Rafelski, Jeremiah Birrell, Andrew Steinmetz, Cheng Tao Yang}
\signature{Johann Rafelski\\ Jeremiah Birrell\\ Andrew Steinmetz\\ Cheng Tao Yang}

\address
{
    Department of Physics,\\
    The University of Arizona,\\
    Tucson, AZ 85721, USA
}

\longindentation=0pt
\begin{document}

\begin{letter}{MDPI Universe}

\opening{Dear Editors,}

Thank you for reviewing our submission to Dr. Remo Ruffini's Festschrift. With this letter, we submit our revised manuscript with changes requested by the reviewers and minor changes we ourselves wished to make. Within the revised manuscript, added or revised text has been highlighted in blue. Minor grammatical changes or corrections remain unmarked. Below we provide a list of relevant changes with the Section denoted:

\begin{itemize}
    \item Sect. 1.2 - Introductory paragraphs revised to discuss the origin of baryon asymmetry and the application of the Sakharov conditions in more detail as recommended by Reviewers \#1 and \#2. Possibilities for contemporary antimatter domains (compact or large) are briefly mentioned.
    \item Sect. 1.3 - Added references to dynamical or phantom dark energy which are alternatives to the $\Lambda$ DE model as recommended by Reviewer \#2. Revised sentences discussing the Hubble tension to better connect to content discussed within this work.
    \item Sect. 3.2 - Emphasized mesons as a critical participant in antimatter evolution as requested by Reviewer \#1.
    \item Sect. 4.1 - Added note that chirality and helicity states are equivalent for fermions in the massless limit.
    \item Sect. 4.2 - Introductory paragraph rewritten for clarity emphasizing overlap between hadronic and leptonic plasma epochs. Section is revised to emphasize antimatter presence in (anti)muon production and decay as recommended by Reviewer \#1.
    \item Sect. 4.3 - The relationship between (anti)matter and Dirac or Majorana neutrinos is expanded as recommended by Reviewer \#1. Influence on the dark matter sector is briefly discussed.
    \item Sect. 4.5 - Brief mention of the possible connection between lepton non-conservation and baryon-antibaryon asymmetry as recommended by Reviewer \#1.
    \item Sect. 5.2 - Emphasis is added that the unique situation of hot dense matter and antimatter in large quantities makes the $e^{\pm}$ epoch uniquely interesting in terms of magnetization as recommended by Reviewer \#1. Classical description of anomalous magnetic moments are briefly mentioned in comparison to quantum description.
    \item Sect. 5.5 - Final paragraph revised to clarify lack of sensitivity in electron-positron chemical potential on magnetic fields within the ranges considered.
    \item Sect. 5.6 - Paragraph expanded to explain the mean field approximation used to obtain magnetization. Emphasis is placed on the connection between the magnetic effects and the large quantities of antimatter present.
    \item Sect. 6 - The photographs honoring Remo Ruffini and Lizhi Fang have been grouped together and moved to the acknowledgements in coordination with editors. The first line of the abstract, as well as the last paragraph of the conclusions have also been moved to acknowledgements with slight revision.
\end{itemize}

The following figures were changed or modified:
\begin{itemize}
    \item Fig. 16 - Caption has been expanded to explain all curves.
    \item Fig. 19 - Figure has been replaced with an corrected value for the Solar core density.
    \item Fig. 23 - Caption rewritten for clarity.
\end{itemize}

The following references were added to support above revisions or otherwise support statements already present in the work:
\begin{itemize}
    \item K.~Eguchi {\em et~al.}, ``{First results from KamLAND: Evidence for reactor anti-neutrino disappearance},'' {\em Phys. Rev. Lett.}, vol.~90, p.~021802, 2003.
    \item G.~L. Fogli, E.~Lisi, A.~Marrone, and A.~Palazzo, ``{Global analysis of three-flavor neutrino masses and mixings},'' {\em Prog. Part. Nucl. Phys.}, vol.~57, pp.~742--795, 2006.
    \item Y.~Fukuda {\em et~al.}, ``{Evidence for oscillation of atmospheric neutrinos},'' {\em Phys. Rev. Lett.}, vol.~81, pp.~1562--1567, 1998.
    \item S.~F. King and C.~Luhn, ``{Neutrino Mass and Mixing with Discrete Symmetry},''{\em Rept. Prog. Phys.}, vol.~76, p.~056201, 2013.
    \item E.~Fernandez-Martinez, J.~Hernandez-Garcia, and J.~Lopez-Pavon, ``{Global constraints on heavy neutrino mixing},'' {\em JHEP}, vol.~08, p.~033, 2016.
    \item S.~Pascoli, S.~T. Petcov, and A.~Riotto, ``{Leptogenesis and Low Energy CP Violation in Neutrino Physics},'' {\em Nucl. Phys. B}, vol.~774, pp.~1--52, 2007.
    \item M.~D. Schwartz, {\em Quantum field theory and the standard model}. Cambridge university press, 2014.
    \item H.~Fritzsch, ``{Neutrino Masses and Flavor Mixing},'' {\em Mod. Phys. Lett. A}, vol.~30, no.~16, p.~1530012, 2015.
    \item H.~Fritzsch and Z.-z. Xing, ``{Mass and flavor mixing schemes of quarks and leptons},'' {\em Prog. Part. Nucl. Phys.}, vol.~45, pp.~1--81, 2000.
    \item C.~Giunti and C.~W. Kim, {\em Fundamentals of neutrino physics and astrophysics}. Oxford university press, 2007.
    \item J.~A. Casas and A.~Ibarra, ``{Oscillating neutrinos and $\mu \to e, \gamma$},'' {\em Nucl. Phys. B}, vol.~618, pp.~171--204, 2001.
    \item N.~Arkani-Hamed, S.~Dimopoulos, G.~R. Dvali, and J.~March-Russell, ``{Neutrino masses from large extra dimensions},'' {\em Phys. Rev. D}, vol.~65, p.~024032, 2001.    
    \item J.~R. Ellis and S.~Lola, ``{Can neutrinos be degenerate in mass?},'' {\em Phys. Lett. B}, vol.~458, pp.~310--321, 1999.
    \item F.~T. Avignone, III, S.~R. Elliott, and J.~Engel, ``{Double Beta Decay, Majorana Neutrinos, and Neutrino Mass},'' {\em Rev. Mod. Phys.}, vol.~80, pp.~481--516, 2008.
    \item I.~Esteban, M.~C. Gonzalez-Garcia, M.~Maltoni, T.~Schwetz, and A.~Zhou, ``{The fate of hints: updated global analysis of three-flavor neutrino oscillations},'' {\em JHEP}, vol.~09, p.~178, 2020.
    \item B.~Abi {\em et~al.}, ``{Deep Underground Neutrino Experiment (DUNE), Far Detector Technical Design Report, Volume II: DUNE Physics}.'' 2 2020.
    \item L.~Alvarez-Ruso {\em et~al.}, ``{NuSTEC White Paper: Status and challenges of neutrino\textendash{}nucleus scattering},'' {\em Prog. Part. Nucl. Phys.}, vol.~100, pp.~1--68, 2018.
    \item J.~Rafelski, M.~Formanek, and A.~Steinmetz, ``{Relativistic Dynamics of Point Magnetic Moment},'' {\em Eur. Phys. J. C}, vol.~78, no.~1, p.~6, 2018.
    \item M.~Formanek, A.~Steinmetz, and J.~Rafelski, ``{Motion of classical charged particles with magnetic moment in external plane-wave electromagnetic fields},'' {\em Phys. Rev. A}, vol.~103, no.~5, p.~052218, 2021.
    \item M.~Formanek, A.~Steinmetz, and J.~Rafelski, ``{Radiation reaction friction: Resistive material medium},'' {\em Phys. Rev. D}, vol.~102, no.~5, p.~056015, 2020.
    \item Rubakov, V.A.; Shaposhnikov, M.E. {Electroweak baryon number nonconservation in the early universe and  in high-energy collisions}. {\em Usp. Fiz. Nauk} {\bf 1996}, {\em 166},~493--537.
    \item Affleck, I.; Dine, M. {A New Mechanism for Baryogenesis}. {\em Nucl. Phys. B} {\bf 1985}, {\em 249},~361--380.
    \item Khlopov, M.Y.; Lecian, O.M. {The Formalism of Milky-Way Antimatter-Domains Evolution}. {\em Galaxies} {\bf 2023}, {\em 11},~50.
    \item M.~Y.~Khlopov, S.~G.~Rubin and A.~S.~Sakharov, {Possible origin of antimatter regions in the baryon dominated universe}, Phys. Rev. D 62, 083505. 2000.
    \item A.~G.~Cohen, A.~De Rujula and S.~L.~Glashow, {A Matter - antimatter universe?}, Astrophys. J. 495, 539-549. 1998.
    \item S.~I.~Blinnikov, A.~D.~Dolgov and K.~A.~Postnov, {Antimatter and antistars in the universe and in the Galaxy}, Phys. Rev. D 92, no.2, 023516. 2015.
    \item Caldwell, R.R.; Kamionkowski, M.; Weinberg, N.N. {Phantom energy and cosmic doomsday}. {\em Phys. Rev. Lett.} {\bf 2003}, {\em 91},~071301.
    \item Bilic, N.; Tupper, G.B.; Viollier, R.D. {Unification of dark matter and dark energy: The Inhomogeneous Chaplygin gas}. {\em Phys. Lett. B} {\bf 2002}, {\em 535},~17--21.
    \item Benevento, G.; Hu, W.; Raveri, M. {Can Late Dark Energy Transitions Raise the Hubble constant?} {\em Phys. Rev. D} {\bf 2020}, {\em 101},~103517.
    \item A.~Biswas, D.~K.~Ghosh and D.~Nanda, {Concealing Dirac neutrinos from cosmic microwave background}, JCAP 10, 006. 2022.
    \item K.~N.~Abazajian and J.~Heeck, {Observing Dirac neutrinos in the cosmic microwave background}, Phys. Rev. D 100, 075027. 2019.
\end{itemize}

The following references were removed:
\begin{itemize}
    \item Yang, C.T.; Rafelski, J. {Bottom quark chemical nonequilibrium in primordial QGP}. Update in preparation, {\bf 2023}.
    \item Demia\'nski, M.; Doroshkevich, A.G. {Beyond the standard $\Lambda$CDM cosmology: the observed structure of DM halos and the shape of the power spectrum}, arXiv:astro-ph.CO/1511.07989. {\bf 2015}.
\end{itemize}
The first removed reference (Yang, 2023, in preparation) was a duplicate of another article included in the citations which is on arXiv, and will be submitted for publication once updated. The second (Demia\'nski, 2015) was replaced with a more relevant reference which has been fully published as requested by Reviewer \#2. Additionally, a note was added in the bibliography justifying the arXiv reference (Fromerth and Rafelski, 2002) which was published in part in Acta Phys. Polon. B (Fromerth et. al., 2012) and in full in Eur. Phys. J. ST (Rafelski, 2019) as requested by Reviewer \#2.


We look forward to having our work accepted for publication.

\closing{Sincerely,}

\end{letter}
\end{document}
